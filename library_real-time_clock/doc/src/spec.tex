%%%%%%%%%%%%%%%%%%%%%%%%%%%%%%%%%%%%%%%%%%%%%%%%%%%%%%%%%%%%%%%%%%%%%%%%%%%
%%
%% Filename: 	spec.tex
%%
%% Project:	A Wishbone Controlled Real-Time clock Core
%%
%% Purpose:	This LaTeX file contains all of the documentation/description
%%		currently provided with this FPGA Real-time Clock Core.
%%		It's not nearly as interesting as the PDF file it creates,
%%		so I'd recommend reading that before diving into this file.
%%		You should be able to find the PDF file in the SVN distribution
%%		together with this PDF file and a copy of the GPL-3.0 license
%%		this file is distributed under.  If not, just type 'make'
%%		in the doc directory and it (should) build without a problem.
%%		
%%
%% Creator:	Dan Gisselquist
%%		Gisselquist Technology, LLC
%%
%%%%%%%%%%%%%%%%%%%%%%%%%%%%%%%%%%%%%%%%%%%%%%%%%%%%%%%%%%%%%%%%%%%%%%%%%%%
%%
%% Copyright (C) 2015, Gisselquist Technology, LLC
%%
%% This program is free software (firmware): you can redistribute it and/or
%% modify it under the terms of  the GNU General Public License as published
%% by the Free Software Foundation, either version 3 of the License, or (at
%% your option) any later version.
%%
%% This program is distributed in the hope that it will be useful, but WITHOUT
%% ANY WARRANTY; without even the implied warranty of MERCHANTIBILITY or
%% FITNESS FOR A PARTICULAR PURPOSE.  See the GNU General Public License
%% for more details.
%%
%% You should have received a copy of the GNU General Public License along
%% with this program.  (It's in the $(ROOT)/doc directory, run make with no
%% target there if the PDF file isn't present.)  If not, see
%% <http://www.gnu.org/licenses/> for a copy.
%%
%% License:	GPL, v3, as defined and found on www.gnu.org,
%%		http://www.gnu.org/licenses/gpl.html
%%
%%
%%%%%%%%%%%%%%%%%%%%%%%%%%%%%%%%%%%%%%%%%%%%%%%%%%%%%%%%%%%%%%%%%%%%%%%%%%%
\documentclass{gqtekspec}
\project{Real-Time Clock}
\title{Specification}
\author{Dan Gisselquist, Ph.D.}
\email{dgisselq (at) opencores.org}
\revision{Rev.~0.1}
\begin{document}
\pagestyle{gqtekspecplain}
\titlepage
\begin{license}
Copyright (C) \theyear\today, Gisselquist Technology, LLC

This project is free software (firmware): you can redistribute it and/or
modify it under the terms of  the GNU General Public License as published
by the Free Software Foundation, either version 3 of the License, or (at
your option) any later version.

This program is distributed in the hope that it will be useful, but WITHOUT
ANY WARRANTY; without even the implied warranty of MERCHANTIBILITY or
FITNESS FOR A PARTICULAR PURPOSE.  See the GNU General Public License
for more details.

You should have received a copy of the GNU General Public License along
with this program.  If not, see \hbox{<http://www.gnu.org/licenses/>} for a
copy.
\end{license}
\begin{revisionhistory}
0.1 & 5/25/2015 & Gisselquist & First Draft \\\hline
\end{revisionhistory}
% Revision History
% Table of Contents, named Contents
\tableofcontents
% \listoffigures
\listoftables
\begin{preface}
Every FPGA project needs to start with a very simple core.  Then, working
from simplicity, more and more complex cores can be built until an eventual
application comes from all the tiny details.

This real time clock is one such simple core.  All of the pieces to this
clock are simple.  Nothing is inherently complex.  However, placing this
clock into a larger FPGA structure requires a Wishbone bus, and being able
to command and control an FPGA over a wishbone bus is an achievement in
itself.  Further, the clock produces seven segment display output values
and LED output values.  These are also simple outputs, but still take a lot
of work to complete.  Finally, this clock will strobe an interrupt line.
Reading and processing that interrupt line requires a whole 'nuther bit of
logic and the ability to capture, recognize, and respond to interrupts.
Hence, once you get a simple clock working, you have a lot working.
\end{preface}

\chapter{Introduction}
\pagenumbering{arabic}
\setcounter{page}{1}

This Real--Time Clock implements a twenty four hour clock, count-down timer,
stopwatch
and alarm.  It is designed to be configurable to adjust to whatever clock
speed the underlying architecture is running on, so with only minor changes
should run on any fundamental clock rate from about 66~kHz on up to
250~TeraHertz with varying levels of accuracy along the way.  

This clock offers a fairly full feature set of capability: time, alarms,
a countdown timer and a stopwatch, all features which are available from the
wishbone bus.

Other interfaces exist as well. 

Should you wish to investigate your clock's
stability or try to guarantee it's fine precision accuracy, it is possible to
provide a time hack pulse to the clock and subsequently read what all of the
internal registers were set to at that time.

When either the count--down timer reaches zero or the clock reaches the alarm
time (if set), the clock module will produce an impulse which can be used
as an interrupt trigger.

This clock will also provide outputs sufficient to drive an external seven
segment display driver and 16 LED's.

Future enhancements may allow for button control and fine precision clock
adjustment.

The layout of this specification follows the format set by OpenCores.
This introduction is the first chapter.  Following this introduction is
a short chapter describing how this clock is implemented,
Chapt.~\ref{chap:arch}.  Following this description, the Chapt.~\ref{chap:ops}
gives a brief overview of how to operate the clock.  Most of the details,
however, are in the registers and their definitions.  These you can find in
Chapt.~\ref{chap:regs}.  As for the wishbone, the wishbone spec requires a
wishbone datasheet which you can find in Chapt.~\ref{chap:wishbone}.
That leaves the final pertinent information necessary for implementing this
core in Chapt.~\ref{chap:ioports}, the definitions and meanings of the
various I/O ports.

As always, write me if you have any questions or problems.

\chapter{Architecture}\label{chap:arch}

Central to this real time clock architecture is a 48~bit sub--second register.
This register is incremented every clock by a user defined 32~bit value,
{\tt CKSPEED}.
When the register turns over at the end of each second, a second has taken
place and all of the various clock registers are adjusted.

Well, not quite but almost.  The 48~bit register is actually split into a
lower 40~bit register that is common to all clock components, as well as
separate eight bit upper registers for the clock, timer, and stopwatch.  In
this fashion, these separate components can have different definitions for
when seconds begin and end, and with sufficient precision to satisfy most
applications.

The next thing to note about this architecture is the format of the various
clock registers: Binary Coded Decimal, or BCD.  Hence an {\tt 8'h59} refers
to a value of 59, rather than 89.  In this fashion, setting the time to
{\tt 24'h231520} will set it to 23~hours, 15~minutes, and 20~seconds.  The
only exception to this BCD format are the subseconds fields found in the
stopwatch and time hack registers.  Seconds and above are all encoded as BCD.

\chapter{Operation}\label{chap:ops}

\section{Time}
To set the time, simply write to the clock register the current value of the
time.  If the seconds hand is written as zero, subsecond time will be cleared
as well.  The new clock value takes place one clock period after the value
is written to the bus.

To set only some parts of the time and not others, such as the minutes but
not seconds or hours, write all '1's to the seconds and hours.  In this way,
writing a {\tt 24'h3f17ff} will set the minutes to 17, but not affect the
rest of the clock.

This is also the way to adjust the display without adjusting time.  Suppose
you wish to switch to display option '1', just write a {\tt 32'h013fffff} to
the register and the display will switch without adjusting time.

\section{Count-down Timer}
To use the count down timer, set it to the amount of time you wish to count
down for.  When ready, or even in the same cycle, enable the count--down
timer by setting the RUN bit high.  At this point in time, the count--down
timer is running.  When it gets to zero, it will stop and trigger an interrupt.
You can tell if the alarm has been triggered by the TRIGGER bit being set.
Any write to the timer register will clear the alarm condition.

While the timer is running, writing a '0' to the timer register will stop it
without clearing the time remaining.  In this state, writing to the register
the RUN bit by itself will restart the timer, while anything else will set the
timer to a new value.  Further, if the timer is stopped at zero, then writing
zero to the timer will reset the timer to the last start time it had.

\section{Stopwatch}
The stop watch supports three operations: start, stop, and clear.  Writing a
'1' to the stop watch register will start the stopwatch, while writing a '0'
will stop it.  When it starts next, it will start where it left off unless the
register is cleared.  To clear the register and set it back to zero, write a
'2' to the register.  This will effectively stop the register and clear it in
one step.  If the register is already stopped, writing a '3' will clear and
start it in one step.  However, the register can only be cleared while stopped.
If the register is running, writing a '3' will have no effect.

\section{Alarm}
To set the alarm, just write the alarm time to the alarm register together
with alarm enable bit.  As with the time register, setting any field,
whether hours, minutes, or seconds, to {\tt 8'hff} has no effect on that
field.  Hence, the alarm may be activated by writing {\tt 25'h13fffff} to
the register and deactivated by writing {\tt 25'h03fffff}.

Once the alarm is tripped, the RTC core will generate an interrupt.  Further,
the tripped bit in the alarm register will be set.  To clear this bit and the
alarm tripped condition, either disable the alarm or write a '1' to this bit.

\section{Time Hacks}

For finer precision timing, the RTC module allows for setting a time
hack and reading the value from the device.  On the clock following the
time hack being high, the internal state, to include the time and the 48~bit
counter, will be recorded and may then be read out.  In this fashion,
it is possible to capture, with as much precision as the device offers,
the current time within the device.

It is the users responsibility to read the time hack registers before a
subsequent time hack pulse sets them to new values.

\chapter{Registers}\label{chap:regs}
This RTC clock module supports eight registers, as listed in
Tbl.~\ref{tbl:reglist}.  Of these eight, the first four have been so placed
as to be the more routine or user used registers, while the latter four are
more lower level.
\begin{table}[htbp]
\begin{center}
\begin{reglist}
CLOCK	& 0 & 32 & R/W & Wall clock time register\\\hline
TIMER	& 1 & 32 & R/W & Count--down timer\\\hline
STPWTCH	& 2 & 32 & R/W & Stopwatch control and value\\\hline
ALARM	& 3 & 32 & R/W & Alarm time, and setting\\\hline\hline
CKSPEED	& 4 & 32 & R/W & Clock speed control.\\\hline
HACKTIME &5 & 32 & R & Wall clock time at last hack.\\\hline
HACKCNTHI&6 & 32 & R & Wall clock time.\\\hline
HACKCNTLO&7 & 32 & R & Wall clock time.\\\hline
\end{reglist}\caption{List of Registers}\label{tbl:reglist}
\end{center}\end{table}
Each register will be discussed in detail in this chapter.

\section{Clock Time Register}
The various bit fields associated with the current time may be found in
the {\tt CLOCK} register, shown in Tbl.~\ref{tbl:clockreg}.
\begin{table}[htbp]\begin{center}
\begin{bitlist}
28--31 & R & Always return zero.\\\hline
24--27 & R/W & Seven Segment Display Mode.\\\hline
22--23 & R & Always return zero.\\\hline
16--21 & R/W & Current time, BCD hours\\\hline
8--15 & R/W & Current time, BCD minutes\\\hline
0--7 & R/W & Current time, BCD seconds\\\hline
\end{bitlist}
\caption{Clock Time Register Bit Definitions}\label{tbl:clockreg}
\end{center}\end{table}
This register contains six clock digits: two each for hours, minutes, and
seconds.  Each of these digits is encoded in Binary Coded Decimal (BCD).
Therefore, 23~hours would be encoded as 6'h23 and not 6'h17.  Writes to each
of the various subcomponent registers will set that register, unless the
write value is a 8'hff.  The behaviour of the clock when non--decimal
values are written, other than all F's, is undefined.

Separate from the time, however, is the seven segment display mode.  Four
values are currently supported: 4'h0 to display the hours and minutes,
4'h1 to display the timer in minutes and seconds, 4'h2 to display the
stopwatch in lower order minutes, seconds, and sixteenths of a second, and
4'h3 to display the minutes and seconds of the current time.  In all cases,
the decimal point will appear to the right of the lowest order digit
and will blink with the second hand.  That is, the decimal will be high for
the second half of any second, and low at the top of the second.

\section{Countdown Timer Register}
The countdown timer register, whose bit--wise values are shown in
Tbl.~\ref{tbl:timer},
\begin{table}[htbp]
\begin{center}
\begin{bitlist}
26--31 & R & Unused, always read as '0'.\\\hline
25 & R/W & Alarm condition.  Write a '1' to clear.\\\hline
24 & R/W & Running, stopped on '0'\\\hline
16--23 & R/W & BCD Hours\\\hline
8--15 & R/W & BCD Minutes\\\hline
0--7 & R/W & BCD Seconds\\\hline
\end{bitlist}
\caption{Count--down Timer register}\label{tbl:timer}
\end{center}\end{table}
controls the operation of the count--down timer.  To use this timer, write
some amount of time to the register, then write zeros with bit 24 set.  The
register will then reach an alarm condition after counting down that amount
of time.  (Alternatively, you could set bit 24 while writing the register,
to set and start it in one operation.)  To stop the register while it is
running, just write all zeros.  To restart the register, provided more than a
second remains, write a {\tt 26'h1000000} to set it running again.  Once
the timer alarms, the timer will stop and the alarm condition will be set.
Any write to the timer register after the alarm condition has been set will
clear the alarm condition.

\section{Stopwatch Register}
The various bits of the stopwatch register are shown in
Tbl.~\ref{tbl:stopwatch}.
\begin{table}[htbp]
\begin{center}
\begin{bitlist}
24--31 & R & Hours\\\hline
16--23 & R & Minutes\\\hline
8--15 & R & Sub Seconds\\\hline
1--7 & R & Sub Seconds\\\hline
1 & W & Clear\\\hline
0 & R/W & Running\\\hline
\end{bitlist}
\caption{Stopwatch Register}\label{tbl:stopwatch}
\end{center}\end{table}
Of note is the bottom bit that, when set, means the stop watch is running.
Set this bit to '1' to start the stopwatch, or to '0' to stop the stopwatch.
Further, while the stopwatch is stopped, a '1' can be written to the clear
bit.  This will zero out the stopwatch and set it back to zero.

\section{Alarm Register}
The various bits of the alarm register are shown in Tbl.~\ref{tbl:alarm}.
\begin{table}[htbp]
\begin{center}
\begin{bitlist}
26--31 & R & Always reads zeros. \\\hline
25 & R/W & Alarm tripped.  Write a '1' to this register to clear any alarm
	condition.  (A tripped alarm will not trip again.)\\\hline
24 & R/W & Alarm enabled\\\hline
16--23 & R & Alarm time, BCD hours\\\hline
8--15 & R & Alarm time, BCD minutes\\\hline
0--7 & R/W & Alarm time, BCD Seconds\\\hline
\end{bitlist}
\caption{Alarm Register}\label{tbl:alarm}
\end{center}\end{table}
Basically, the alarm register consists a time and two more bits.  The extra
two bits encode whether or not the alarm is enabled, and whether or not it has
been tripped.  The alarm will be {\em tripped} whenever it is enabled, and the
time changes to equal the alarm time.  Once tripped, the alarm will stay
in the alarmed or tripped condition until either a '1' is written to the 
tripped bit, or the alarm is disabled.

As with the clock and timer registers, writing eight ones to any of the
BCD fields when writing to this register will leave those fields untouched.

\section{Clock Speed Register}
The actual speed of the clock is controlled by the {\tt CKSPEED} register, 
shown in Tbl.~\ref{tbl:ckspeed}. 
\begin{table}[htbp]
\begin{center}
\begin{bitlist}
0--31 & R/W & 48~bit counter time increment\\\hline
\end{bitlist}
\caption{Clock Speed Register}\label{tbl:ckspeed}
\end{center}\end{table}
This register contains a simple 32~bit unsigned value.  To step the clock,
this value is extended to 48~bits and added to the fractional seconds value.

This value should be set to $2^{48}$ divided by the clock frequency of the
controlling clock.  Hence, for a 100~MHz clock, this value would be set to
{\tt 32'd2814750}.  For clocks near 100~MHz, this allows adjusting speed
within about 40~clocks per second.  For clocks near 500~MHz, this allows
time adjustment to an accuracy of about about 800~clocks per second.  In
both cases, this is good enough to maintain a clock with less than a
microsecond loss over the course of a year.  Hence, this RTC module provides
more logical stability than most hardware clocks on the market today.

\section{Time--hack time}
To support finer precision clock control, the time--hack capability exists.
This capability consists of three registers, the time--hack time register
shown in Tbl.~\ref{tbl:hacktime},
\begin{table}[htbp]
\begin{center}
\begin{bitlist}
24--31 & R & BCD Hours.\\\hline
16--23 & R & BCD Minutes.\\\hline
8--15 & R & BCD seconds.\\\hline
0--7 & R & Subseconds, encoded in 256ths of a second\\\hline
\end{bitlist}
\caption{Time Hack Time Register}\label{tbl:hacktime}
\end{center}\end{table}
and two registers (Tbls.~\ref{tbl:hackcnthi}
\begin{table}[htbp]
\begin{center}
\begin{bitlist}
0--31 & R & Upper 32 bits of the internal 40~bit counter.\\\hline
\end{bitlist}
\caption{Time Hack Counter, High}\label{tbl:hackcnthi}
\end{center}\end{table}
and~\ref{tbl:hackcntlo})
\begin{table}[htbp]
\begin{center}
\begin{bitlist}
24--31 & R & Bottom 8~bits of the internal 40~bit counter.\\\hline
0--23 & R & Always read as '0'.\\\hline
\end{bitlist}
\caption{Time Hack Counter, Low}\label{tbl:hackcntlo}
\end{center}\end{table}
capturing the contents of the 40~bit internal counter at the time of the hack.

The time--hack time register is perhaps the simplest to understand.  This 
captures the time of the time--hack in hours, minutes, seconds, and 8~fractional
subsecond bits.  The top 24~bits of this register will match the bottom 24~bits
of the clock~time register at the time of the time hack.  The bottom eight
bits are the top eight bits of the 48~bit subsecond time counter.  The
rest of those 48~bits may then be returned in the other two time hack counter
registers.

At present, this functionality isn't yet truly fully featured.  Once fully
featured, there will (should) be a mechanism for adjusting this counter based
upon information gleaned from the hack time.  Implementation details have
to date prevented this portion of the design from being implemented.

\chapter{Wishbone Datasheet}\label{chap:wishbone}
Tbl.~\ref{tbl:wishbone}
\begin{table}[htbp]
\begin{center}
\begin{wishboneds}
Revision level of wishbone & WB B4 spec \\\hline
Type of interface & Slave, Read/Write \\\hline
Port size & 32--bit \\\hline
Port granularity & 32--bit \\\hline
Maximum Operand Size & 32--bit \\\hline
Data transfer ordering & (Irrelevant) \\\hline
Clock constraints & Faster than 66~kHz \\\hline
Signal Names & \begin{tabular}{ll}
		Signal Name & Wishbone Equivalent \\\hline
		{\tt i\_clk} & {\tt CLK\_I} \\
		{\tt i\_wb\_cyc} & {\tt CYC\_I} \\
		{\tt i\_wb\_stb} & {\tt STB\_I} \\
		{\tt i\_wb\_we} & {\tt WE\_I} \\
		{\tt i\_wb\_addr} & {\tt ADR\_I} \\
		{\tt i\_wb\_data} & {\tt DAT\_I} \\
		{\tt o\_wb\_ack} & {\tt ACK\_O} \\
		{\tt o\_wb\_stall} & {\tt STALL\_O} \\
		{\tt o\_wb\_data} & {\tt DAT\_O}
		\end{tabular}\\\hline
\end{wishboneds}
\caption{Wishbone Datasheet}\label{tbl:wishbone}
\end{center}\end{table}
is required by the wishbone specification, and so 
it is included here.  The big thing to notice is that this real time clock
acts as a wishbone slave, and that all accesses to the
clock registers are 32--bit reads and writes.  The address bus does not offer
byte level, but rather 32--bit word level resolution.  Select lines are not
implemented.  Bit ordering is the normal ordering where bit~31 is the most
significant bit and so forth.  Although the stall line is implemented, it is
always zero.  Access delays are a single clock, so the clock after a read or
write is placed on the bus the {\tt i\_wb\_ack} line will be high.

\iffalse
\chapter{Clocks}\label{chap:clocks}

This core is based upon the Basys--3 design.  The Basys--3 development board
contains one external 100~MHz clock, which is sufficient to run this
core.  The logic within the core can also be run faster, or slower, as is
necessary to meet the timing constraints associated with the internal
operations of the core and it's surrounding environment.  See
Table.~\ref{tbl:clocks}.
\begin{table}[htbp]
\begin{center}
\begin{clocklist}
i\_clk & External & 250~THz & 66~kHz & System clock.\\\hline
\end{clocklist}
\caption{List of Clocks}\label{tbl:clocks}
\end{center}\end{table}

\fi

\chapter{I/O Ports}\label{chap:ioports}
The I/O ports for this core are shown in Tbls.~\ref{tbl:iowishbone}
\begin{table}[htbp]
\begin{center}
\begin{portlist}
i\_wb\_cyc & 1 & Input & Wishbone bus cycle wire.\\\hline
i\_wb\_stb & 1 & Input & Wishbone strobe.\\\hline
i\_wb\_we & 1 & Input & Wishbone write enable.\\\hline
i\_wb\_addr & 5 & Input & Wishbone address.\\\hline
i\_wb\_data & 32 & Input & Wishbone bus data register for use when writing
	(configuring) the core from the bus.\\\hline
o\_wb\_ack & 1 & Output & Return value acknowledging a wishbone write, or
		signifying valid data in the case of a wishbone read request.
		\\\hline
o\_wb\_stall & 1 & Output & Indicates the device is not yet ready for another
		wishbone access, effectively stalling the bus.\\\hline
o\_wb\_data & 32 & Output & Wishbone data bus, returning data values read
		from the interface.\\\hline
\end{portlist}
\caption{Wishbone I/O Ports}\label{tbl:iowishbone}
\end{center}\end{table}
and~Tbl.~\ref{tbl:ioother}. 
\begin{table}[htbp]
\begin{center}
\begin{portlist}
o\_sseg & 32 & Output & Lines to control a seven segment display, to be
		sent to that display's driver.  Each eight bit byte controls
		one digit in the display, with the bottom bit in the byte
		controlling the decimal point.\\\hline
o\_led & 16 & Output & Output LED's, consisting of a 16--bit counter counting
		from zero to all ones each minute, and synchronized with each
		minute so as to create an indicator of when the next minute
		will take place when only the hours and minutes can be
		displayed.\\\hline
o\_interrupt & 1 & Output & A pulsed/strobed interrupt line.  When the
		clock needs to generate an interrupt, it will set this line
		high for one clock cycle.  \\\hline
i\_hack & 1 & Input & When this line is raised, copies are made of the
	internal state registers on the next clock.  These registers can then
	be used for an accurate time hack regarding the state of the clock
	at the time this line was strobed.\\\hline
\end{portlist}
\caption{Other I/O Ports}\label{tbl:ioother}
\end{center}\end{table}
Tbl.~\ref{tbl:iowishbone} reiterates the wishbone I/O values just discussed in
Chapt.~\ref{chap:wishbone}, and so need no further discussion here.

This clock is designed for command and control via the wishbone.  No other
registers, beyond the wishbone bus, are required.  However, several other
may be valuable.  These other registers are listed in Tbl.~\ref{tbl:ioother}. 
We'll discuss each of these in turn.

First of the other I/O registers is the {\tt o\_sseg} register.  This register
encodes which outputs of a seven segment display need to be turned on to
represent the value of the clock requested.  This register consists of four
eight bit bytes, with the highest order byte referencing the highest order
display segment value.  In each byte, the low order bit references a decimal
point.  The other bits are ordered around the zero, with the top bit being
the top bar of a '0', the next highest order bit and so on following the
zero clockwise.  The final bit of each byte, the bit in the two's place, 
encodes whether or not the middle line is to be displayed.  When either timer
or alarm is triggered, this display will blink until the triggering conditions
are cleared.

This output is expected to be the input to a seven segment display driver,
rather than being the output to the display itself.

The next output lines are the 16~lines of the {\tt o\_led} bus.  When connected
with 16~LED's, these lines will create a counting display that will count up
to each minute, synchronized to the minute.  When either timer or alarm has
triggered, all of the LED's will flash together until the triggered condition
is reset.

The third other line is the {\tt o\_interrupt} line.  This line will be 
strobed by the RTC module any time the alarm is triggered or the timer runs
out.  The line will not remain high, but neither will it trigger a second
time until the underlying interrupt is cleared.  That is, the timer will only
trigger once until cleared as will the alarm, but the alarm may trigger after
the timer has triggered and before the timer clears.

The final other I/O line is a simple input line.  This line is expected to be
strobed for one clock cycle any time a time hack is required.  For example,
should you wish to read and synchronize to a GPS PPS signal, strobe the device
with the PPS (after dealing with any metastability issues), and read the time
hacks that are produced.

% Appendices
% Index
\end{document}


