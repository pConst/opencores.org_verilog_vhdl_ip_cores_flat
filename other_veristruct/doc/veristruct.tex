\documentclass[a4paper]{article} \usepackage[dvips]{graphicx}
\usepackage[margin=2cm,nohead]{geometry} \usepackage{pslatex}
\usepackage{listings} 
\setlength{\parindent}{0pt} \setlength{\parskip}{10pt}
\renewcommand{\familydefault}{\sfdefault}
\begin{document}
\author{Michael Cowell} \title{Veristruct: An IEEE1364.1995 (Verilog
  1995) Preprocessor}
\maketitle{}
% \tableofcontents{}
% \pagebreak{}
\section{Overview}
\label{sec:overview}
Veristruct is an an IEEE1364.1995 preprocessor that adds some C-style
struct support to the Verilog language. It takes as input Veristruct
files (with a \texttt{.vs} extension) and struct definition files
(with a \texttt{.struct} extension). It outputs standard Verilog files
(with a \texttt{.v} extension). Veristruct files are, for the most
part, standard Verilog files---but, as well as the normal
\texttt{nets} and \texttt{regs}, bundled, hierarchical variables can
be declared and used.

Veristruct can process one veristruct file per invocation (which may
include many \texttt{.struct} struct definition files).

Veristruct is written in object oriented Perl. Please feel free to fix
any bugs you find!\footnote{That includes those in this
  documentation.}

\section{Installation}
\label{sec:installation}
First you will have to get Veristruct. A tarball of the latest release
should be available at the Rachael SPARC
website\footnote{\texttt{https://www.rachaelsparc.org}}.
Alternatively, you can check out the source directly with a command
like:
$$\hbox{\texttt{svn co 
    https://www.rachaelsparc.org/repository/projects/tools/veristruct/}}$$
Note that code in SVN is not guaranteed to produce correct output or
even run, but you will probably get more features than in the release
tarball.

Next, you need to add the Veristruct classes to your Perl
distribution.  One day I might put Veristruct in CPAN and this process
will be automated, but until then you'll need to do things
manually. First work out where you want to place the classes. They
need to be in your Perl include path. You can either add the folder
that you've got Veristruct in to your Perl include path---or find out
what's in your \texttt{@INC} (run something like \texttt{perl -e
  'print join(", ",@INC);'}) and copy the \texttt{Verilog} folder from
the Veristruct root into one of those directories.

Finally, you need to put the \texttt{veristruct.pl} script (the
application) somewhere in your shell path. Once again you can either
change your shell's \texttt{PATH} variable to point to your veristruct
folder, or copy \texttt{veristruct.pl} to \texttt{/usr/local/bin} or
something. If you're copying it, you might want to strip \texttt{.pl}
off of the name (to have a nicer looking command).

\section{Command-line Options}
\label{sec:options}
The following table describes each of the Veristruct application's
command line arguments:

\begin{center}
  \begin{tabular}{|l|p{10cm}|}
    \hline \textbf{Argument} & \textbf{Action} \\
    \hline \texttt{--help}, \texttt{-h} & Prints usage information. \\
    \hline \texttt{--debug}, \texttt{-d} & Prints debugging information
    while parsing the input files. This is useful if (for instance) Veristruct
    fails to parse a syntactically valid file, and you are interested in what
    mistake it made. Note: this option is intended for developers.\\
    \hline \texttt{--write}, \texttt{-w} & By default Veristruct will not overwrite
    an existing file with the same name as the output file. This options causes
    it to do so.\\
    \hline \texttt{--infile=file}, \texttt{-i file} & This compulsory argument
    should be used to specify the input \texttt{.vs} file. Note: the \texttt{.vs} 
    extension is not compulsory, but is recommended for clarity.\\
    \hline \texttt{--outfilef=file}, \texttt{-o file} & This compulsory argument
    should be used to specify the output \texttt{.v} file. Note: the \texttt{.v}
    extension is not compulsory, but is recommended for clarity.\\
    \hline \texttt{--libpath=path}, \texttt{-L path} & This argument (which can
    be used multiple times) should be used to specify the search path
    used when trying to locate \texttt{.struct} files referenced in the input.
    It can be either a path, or a comma separated list of paths.\\
    \hline
  \end{tabular}
\end{center}

The command line arguments are not case sensitive, and it is
sufficient to specify just enough of the command as is unique
(although this is not recommended, because more commands may be added
in the future).

\section{Input \& Output Files}
\label{sec:files}
Veristruct takes two sorts of input files, \texttt{.vs} (or
Veristruct) files and \texttt{.struct} (or struct definition)
files. Following is a description of the two.

\subsection{Veristruct files --- .vs}
\label{sec:vs_files}
Veristruct files are, for the most part, standard IEEE1364.1995
(Verilog) files.  The following extra syntax is allowed:

\begin{itemize}
\item \textbf{\texttt{`sinclude}}: This extra pre-processor directive
  can be used to instruct Veristruct to read in a struct definition
  file. The file to be read in should follow immediately after the
  token (with possible whitespace, other than line breaks). This
  directive may be used as many times as desired, to read in any
  number of \texttt{.struct} files. Note: the directive should not be
  used inside modules, and surrounding \texttt{`ifdef} directives will
  \textbf{not} be honoured.
\item \textbf{\texttt{reg}, \texttt{wire} and port struct
    declarations}: If any of these Verilog declarations are followed
  by the name of a defined struct, the next word will interpreted as
  an instance of that struct. All uses of that variable, throughout
  the rest of the module (and, in the case of module port lists,
  earlier in the module as well) will be recognised by Veristruct as
  struct references, and will be re-written in the output file. Arrays
  of structs can be declared in the same way that normal vectors are
  declared (the range should precede the struct name).
\item \textbf{Struct element references}: Any expression token (or
  l-value) in the input \texttt{.vs} file of the form
  \texttt{inst\_name.element} (of arbitrary element depth, and with
  optional ranges) are recognised and processed by Veristruct. The
  semantics are the same as in C. Note that slices of struct elements,
  where the struct is declared as an array, are not allowed as
  l-values.
\end{itemize}

Whole structs can only be used as l-values if the expression to the
right is also a whole struct, of the same type, and nothing else.

Wire declarations can include assignments, as usual.

\subsection{Struct declaration files --- .struct}
\label{sec:struct_files}
The EBNF for struct declaration files is:

\texttt{struct\_file := struct\_block \\
  struct\_block := struct\_defn \{ struct\_block \} \\
  struct\_defn := \textbf{struct} struct\_name
  \textbf{\{} elem\_block \textbf{\};} \\
  elem\_block := elem \textbf{;} \{ elem\_block \} \\
  elem := wire\_element | signed\_element | struct\_element \\
  wire\_element := \textbf{wire}((elem\_name \textbf{[} max \textbf{:}
  min \textbf{]})
  | (elem\_name) | (elem\_name \textbf{`} define\_string)) \\
  signed\_element := \textbf{signed}((elem\_name \textbf{[} max
  \textbf{:} min \textbf{]})
  | (elem\_name) | (elem\_name \textbf{`} define\_string)) \\
  struct\_element := struct\_name elem\_name }

The atomic tokens are:
\begin{itemize}
\item \texttt{struct\_name}: A valid Verilog variable name. Trying to
  use ``wire'' (which is illegal anyway) is doubly bad. Double
  underscores are not allowed (this is not actually enforced, but is a
  very bad idea).
\item \texttt{elem\_name}: Similar to above.
\item \texttt{max}: An integer greater than \texttt{min}.
\item \texttt{min}: An integer less than \texttt{max}.
\item \texttt{define\_string}: Any string without whitespace.
\end{itemize}

The semantics are similar to C. Note the main syntactical difference
is that a name does not need to be specified twice (as in C, after the
closing brace). C-style comments are allowed wherever (they are
stripped out prior to parsing). Line comments count as a line break,
block comments count as nothing.

\subsection{IEEE1364.1995 (Verilog) file --- .v}
\label{sec:v_files}
Veristruct outputs IEEE1364.1995 compliant \texttt{.v} files. Structs
declarations are expanded into their elements, and struct element
references are expanded into formally compliant variable names. Double
underscores are used to indicate depth.

When entire structs are referenced (instance names are used without
element references) a guess is made as to what is desired. In
sensitivity lists and module port lists, the name is exploded into its
constituent elements (and appropriate separator tokens are added). In
module instance port lists, it is assumed that the module being
instantiated has multiple ports with the instance name as the base
(that will be the case if the module was written in Veristruct).

\section{Limitations}
\label{sec:limitations}
The following limitations currently exist within Veristruct:
\begin{itemize}
\item \textbf{Namespace collisions not detected}: Veristruct currently
  does no checking for namespace collisions. If you don't use double
  underscores in your variables names, though, you should be fine.
\item \textbf{Limited lvalue support}: Slices of some struct elements
  (and whole structs, in most contexts) cannot be lvalues. Veristruct
  will return an error when it encounters these.
\item \textbf{Structs can only be declared in \texttt{.struct} files}:
  This is a deliberate limitation. It avoids polluting the Verilog
  syntax even more.
\item \textbf{\texttt{`ifdef}s ignored}: Because Veristruct is not
  designed to replace the normal Verilog preprocessor, it does not do
  its job. As such, you will need to ensure that do not rely on your
  structs being conditionally included using normal pre-processor
  directives. There are many ways around this. Ranges in structs can
  be tick defined, and variable declarations can be surrounded in
  \texttt{`ifdef}s, which will get honoured later.
\end{itemize}

\section{Changes}
\label{sec:changes}
\textbf{Changes since the 0.1 release}\footnote{References to issue
  IDs refer to the Rachael SPARC Scarab bug tracking system. Go to
  \texttt{https://www.rachaelsparc.org/scarab/issues} to see
  details.}:
\begin{itemize}
\item Signed support added (as per the request made by luked in
  RCHL10).
\item Bugfixes discussed in RCHL12.
\end{itemize}

\section{Examples}
\label{sec:examples}
\lstset{numbers=left, numberstyle=\tiny, stepnumber=1, numbersep=5pt,
  basicstyle=\ttfamily, keywordstyle=\ttfamily,
  identifierstyle=\ttfamily, commentstyle=\ttfamily,
  stringstyle=\ttfamily, breaklines}

\subsection{Example .struct file}
\label{sec:ex_struct_file}
Here is an example .struct file:

\lstinputlisting{../test.struct}

\subsection{Example .vs file}
\label{sec:ex_vs_file}
And an example .vs file that uses those structs:

\lstinputlisting{../test.vs}

\subsection{Example .v file}
\label{sec:ex_v_file}
Finally, here the the .v file that Veristruct outputs:

\lstinputlisting{../test.v}
\end{document}
