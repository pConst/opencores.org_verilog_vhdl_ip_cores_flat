% The Potato Processor - User's Manual
% (c) Kristian Klomsten Skordal 2015 <skordal@opencores.org>
% Report bugs and issues on <http://opencores.org/project,potato,bugtracker>

\documentclass[12pt,a4paper]{report}

\usepackage[utf8]{inputenc}
\usepackage[pdftitle={The Potato Processor: Technical Reference Manual},
	pdfauthor={Kristian Klomsten Skordal}]{hyperref}
\usepackage{placeins}
\usepackage{titlesec}
\usepackage[english]{babel}

\newcommand{\register}[1]{\textsc{#1}}

\titleformat{\chapter}[block]{\normalfont\huge\bfseries}{\thechapter}{20pt}{\Huge}\titlespacing*{\chapter}{0pt}{50pt}{40pt}

\begin{document}

\begin{titlepage}
	\begin{center}
		\vspace*{3cm}
		{\large The Potato Processor}\\[0.5cm]
		{\Huge\bf Technical Reference Manual}\\[6cm]

		\textsc{Kristian Klomsten Skordal}\\\href{mailto:skordal@opencores.org}{skordal@opencores.org}\\[3em]

		\vfill
		{Project page:\\\url{http://opencores.org/project,potato}}\\[0.2em]
		{Report bugs and issues on\\\url{http://opencores.org/project,potato,bugtracker}}\\[1.2em]
		{\small Updated \today}
	\end{center}
\end{titlepage}

\tableofcontents

\chapter{Introduction}

The Potato processor is an implementation of the 32-bit integer subset of the RISC-V
instruction set v2.0. It is designed around a standard 5-stage pipeline. All instructions
execute in 1 cycle, with the exception of load and store instructions when the processor
has to wait for external memory.

The processor has been tested on an Artix~7 (xc7a100tcsg324-1) FPGA from Xilinx, on the
Nexys 4 board from Digilent. More details about the test design can be found in chapter
\ref{cha:quickstart}.

\section{Features}
Here is a highlight of the current features of the Potato processor:

\begin{itemize}
\item Implements the complete 32-bit integer subset of the RISC-V ISA v2.0.
\item Implements the \textsc{csr*} instructions from the RISC-V supervisor extensions v1.0.
\item Supports using the \register{fromhost}/\register{tohost} registers for communicating
	with a host environment, such as a simulator, or peripherals.
\item Supports exception handling, with support for 8 individually maskable IRQ inputs.
\item Includes a wishbone B4 compatible interface.
\end{itemize}

\section{Planned features}
Here is a highlight of the future planned features of the Potato processor:

\begin{itemize}
\item Caches.
\item Branch prediction.
\item Hardware multiplication and division support (the RISC-V M extension).
\item Compressed instruction support (the RISC-V C extension).
\item Supervisor mode support
\end{itemize}

\chapter{Quick Start Guide}
\label{cha:quickstart}

This chapter contains instructions on getting started with the demo/example design that is
included in the Potato source distribution. The example design targets the Nexys 4 board
available from Digilent\footnote{See \url{http://www.digilentinc.com/Products/Detail.cfm?Prod=NEXYS4}}.

\section{Setting up the Vivado Project}

Start by creating a new project in Vivado. Import all source files from the \texttt{src/} directory,
which contains all source files required for using the processor. Then import all source files from
the \texttt{example/} directory, which contains the toplevel setup for the example SoC design,
and from the \texttt{soc/} directory, which contains various peripherals for the processor.

\section{Adding IP Modules}

The example design requires two additional IP modules. These are not included in the source
distribution and must be added separately.

\subsection{Clock Generator}

Add a clock generator using the Clocking Wizard. Name the component ``\texttt{clock\_generator}''
and make sure that the checkboxes for ``frequency synthesis'' and ``safe clock startup'' are
selected.

Add two output clocks with frequencies of 50~MHz and 10~MHz. Rename the corresponding ports
to ``\texttt{system\_clk}'' and ``\texttt{timer\_clk}''. Rename the input clock signal to
``\texttt{clk}''.

The added module should appear in the hierarchy under the toplevel module as ``\texttt{clkgen}''.

\subsection{Instruction memory}

Add a block RAM to use for storing the test application using the Block Memory Generator.
Choose ``Single-port ROM'' as memory type and name the module ``\texttt{instruction\_rom}''.
Set port A width to 32 bits and the depth to 2048 words. Initialize the block RAM with
your application or use one of the provided benchmarks, such as the SHA256 benchmark,
which, when built, produces a \texttt{.coe} file that can be used for this purpose.

Note that in order to build a benchmark application, you have to install a RISC-V toolchain.
See section \ref{sec:toolchain} for instructions on how to accomplish this.

\section{Running an Example Application}

Assuming you initialized the instruction memory with the SHA256 benchmark, synthesize and
implement the design, generate a bitfile and load it into the FPGA. Using a serial port
application, such as \texttt{minicom}, watch as the number of hashes per second are
printed to the screen and rejoice because it works!

\chapter{Instantiating}

The Potato processor can be used either with or without a wishbone interface. Using the wishbone
interface allows the processor to communicate with other wishbone-compatible peripherals. However,
if no such peripherals are to be used, the processor can, for instance, be connected directly to
block RAM memories for full performance without needing to use caches.

\section{Customizing the Processor Core}
The processor can be customized using generics. The following list details the parameters
that can be changed:

\begin{description}
	\item[\texttt{PROCESSOR\_ID}:] Any 32-bit value used as the processor ID. This value can
	be read back from the hardware thread ID register, \register{hartid}.
	\item[\texttt{RESET\_ADDRESS}:] Any 32-bit value used as the address of the first instruction
	fetched by the processor after it has been reset.
\end{description}

\section{Instantiating in a Wishbone System}
\label{sec:instantiating-wishbone}

In order to integrate the Potato processor into a wishbone-based system, the module \texttt{pp\_potato}
is used. It provides signals for the wishbone master interface, prefixed with \texttt{wb\_}, and
inputs for interrupts and the HTIF interface.

The specifics of the wishbone interface is listed in table \ref{tab:wishbone}. To see an example
of the processor used in a Wishbone system, see the example design under the \texttt{example/}
directory.

\begin{table}
	\centering
	\begin{tabular}{|l|l|}
		\hline
		Wishbone revision		& B4 \\
		Interface type			& Master \\
		Address port width		& 32 bits \\
		Data port width			& 32 bits \\
		Data port granularity		& 8 bits \\
		Maximum operand size		& 32 bits \\
		Endianess			& Little endian \\
		Sequence of data transfer	& Not specified \\
		\hline
	\end{tabular}
	\caption{Wishbone Interface Specifics}
	\label{tab:wishbone}
\end{table}

\FloatBarrier

\section{Instantiating in a Standalone System}
\label{sec:instantiating-standalone}

The processor can also be used without connecting it to the Wishbone bus. An example
of this can be seen in the processor testbench, \texttt{tb\_processor.vhd}.

\section{Verifying}

The processor provides an automatic testing environment for verifying that the processor
correctly executes the instructions of the ISA. The tests have been extracted from the
official test suite available at \url{https://github.com/riscv/riscv-tests} and covers
most of the available instructions.

Two testbenches are used to execute the test programmes: \texttt{tb\_processor.vhd}, in
which the processor is directly connected to block-RAM-like memories so the processor
never stalls to wait for memory operations to finish (see section \ref{sec:instantiating-standalone}
for more details about this kind of setup), and \texttt{tb\_soc.vhd}, which models a
simple system-on-chip with the processor connected to memories through the
wishbone interface (see section \ref{sec:instantiating-wishbone} for more information
about this kind of setup).

To run the test suites, run \texttt{make run-tests} or \texttt{make run-soc-tests}.

Make sure that \texttt{xelab} and \texttt{xsim} is in your \texttt{PATH} or the
tests will fail.

\chapter{Programming}

The processor implements the RISC-V instruction set, and can be programmed with tools
such as GCC.

\section{Building a RISC-V Toolchain}
\label{sec:toolchain}

An ``official'' toolchain is provided by the RISC-V project. In order to install it, clone
the ``riscv-tools'' Git repository from \url{https://github.com/riscv/riscv-tools} and follow
the instructions provided by the README file.

\section{Control and Status Registers}

The supported control and status registers are shown in table \ref{tab:csr_list}. The registers
can be manipulated using the \textsc{csr*} family of instructions, listed in \ref{sec:csr_instrs}.

\begin{table}
	\centering
	\begin{tabular}{|l|l|l|}
		\hline
		\textbf{Name}		& \textbf{ID}	& \textbf{Description} \\
		\hline
		\register{hartid}	& 0x50b		& Hardware thread ID \\
		\register{evec}		& 0x508		& Exception vector address \\
		\register{epc}		& 0x502		& Return address for exceptions \\
		\register{cause}	& 0x509		& Exception cause \\
		\register{sup0}		& 0x500		& Support register 0, for operating system use \\
		\register{sup1}		& 0x501		& Support register 1, for operating system use \\
		\register{badvaddr}	& 0x503		& Bad address, used for invalid address exceptions \\
		\register{status}	& 0x50a		& Processor status and control register \\
		\register{tohost}	& 0x51e		& Register for sending data to a host system \\
		\register{fromhost}	& 0x51f		& Register where data received from a host system is stored \\
		\register{cycle}	& 0xc00		& Cycle counter, low 32 bits \\
		\register{cycleh}	& 0xc80		& Cycle counter, high 32 bits \\
		\register{time}		& 0xc01		& Timer tick counter, low 32 bits \\
		\register{timeh}	& 0xc81		& Timer tick counter, high 32 bits \\
		\register{instret}	& 0xc02		& Retired instruction counter, low 32 bits \\
		\register{instreth}	& 0xc82		& Retired instruction counter, high 32 bits \\
		\hline
	\end{tabular}
	\caption{List of Control and Status Registers}
	\label{tab:csr_list}
\end{table}
\FloatBarrier

\chapter{Instruction Set}

The Potato processor is designed to support the full 32-bit integer subset of
the RISC-V instruction set, version 2.0. The ISA documentation is available
from \url{http://riscv.org}.

\section{Status and Control Register Instructions}
\label{sec:csr_instrs}

In addition to the base ISA, some additional instructions have been imported
from the RISC-V supervisor specification\footnote{The processor is in
the process of being upgraded to the new specification.} version 1.0.

\begin{table}[htb]
	\centering
	\begin{tabular}{|l|l|}
		\hline
		\textbf{Mnemonic}		& \textbf{Description} \\
		\hline
		\texttt{scall}			& System call \\
		\texttt{sbreak}			& Breakpoint instruction \\
		\texttt{sret}			& Exception return \\
		\hline
		\texttt{csrrw rd, rs1, CSR}	& Writes rs1 into CSR, place sold value in rd \\
		\texttt{csrrs rd, rs1, CSR}	& Ors rs1 with CSR, places old value in rd \\
		\texttt{csrrc rd, rs1, CSR}	& Ands the inverse of rs1 with CSR, places old value in rd \\
		\texttt{csrrwi rd, imm, CSR}	& Writes imm into CSR, places old value in rd \\
		\texttt{csrrsi rd, imm, CSR}	& Ors CSR with imm, places old value in rd \\
		\texttt{csrrci rd, imm, CSR}	& Ands the inverse of imm with CSR, places old value in rd \\
		\hline
	\end{tabular}
	\caption{List of CSR Instructions}
\end{table}

\appendix

\chapter{Peripherals}

The source distribution of the processor contains several peripheral modules that can be
used in system-on-chip designs using the Potato processor (or other processors).

This chapter briefly describes each of the modules.

\section{GPIO}

The GPIO module provides a simple GPIO interface for up to 32 general purpose pins.
Each pin can be separately configured to work as either an input or an output pin.

Registers are provided to set the direction of each pin. Additional registers
provide the ability to read or write the values of the pins.

\section{Timer}

The timer module provides a timer that fires off an interrupt at a specified
interval.

\section{UART}

The UART module provies a fixed-baudrate serial port interface. It features
separate FIFOs for buffering input and output data, and interrupts for when
the module is ready to send or has received data.

\section{Memory}

The memory module is basically a simple block RAM wrapper with support for
byte-writes.

\end{document}

